\documentclass{article}

\usepackage[a4paper,margin=2.5cm]{geometry}
\parindent=0pt
\frenchspacing

\usepackage[none]{hyphenat}
\usepackage{parskip}
\usepackage[hyphens]{url}
\usepackage{hyperref}

\hypersetup{colorlinks, citecolor=red, filecolor=red, linkcolor=black, urlcolor=blue}

\begin{document}

\title{Setting up the Zumo robot}
\author{Leon Helwerda}
\date{\today}

\maketitle

\section{Introduction}

The Zumo has documentation that does require some searching in case one wants 
to connect it differently than one usually would (with only an Arduino on the 
Zumo Shield). Since we might want to do certain processing on the Raspberry Pi, 
or maybe even remove the Arduino out of the equation, we need to know which 
pins are used for what specifically and forward information to the Pi. The Zumo 
has a software library, written in C and Arduino code, that offer more details 
about the pin uses. Additionally, documentation and data sheets corresponding 
the individual parts also provide more information.

\section{Pin numbering}
The following pin numbers are useful to know for the Zumo Shield robot. We will 
have to make appropriate connections from an Arduino (which corresponds nicely 
with the shield) or a Raspberry Pi (which then has its own numbering). One can 
also solder an additional array of pins on the shield to forward signals from 
the Arduino to the Pi.

There exists a difference between analog and digital pins in numbering, but as 
far as it seems both can be used for normal purposes.

The following pins are used for various components and interfaces with the Zumo 
Shield and Arduino Uno:

\begin{itemize}
  \item LEDs: two red/blue power LEDS with no pin control, one yellow LED 
        controllable using pin 13.
  \item Push buttons: a reset button that is connected to the RST pin. Usually 
        this is connected to a RESET pin on the Arduino. Note that we can add 
        other peripherals to the RST pin to cause the Arduino to reset, or 
        connect it to a DTR line of a serial connection to reset it from the 
        Raspberry Pi. Be sure that there is not too much power on this line and 
        do not power it when the Zumo itself is not powered, otherwise the 
        brown-out detection will kick in and cause a reset loop. \\
        Aside from the reset button, there is a user button that is connected 
        to pin 12.
  \item Motor: pins 7 (left) and 8 (right) are for direction. A low signal 
        causes a motor to drive forward, and a high signal drives it backward.  
        Pins and pins 9 and 10 are for motor speed. The speeds are sent as PWM 
        values that are between 0 (no throttle) and 255 (maximum speed). The 
        Zumo libraries use 0 to 400 and convert that using a multiplier 
        $51/80$, unless 20KHz PWM is used.
  \item Buzzer: can be used to play simple sounds by passing PWM values. Can be 
        controlled using pin 3 or pin 6, depending on the Arduino being used. 
        By default, the buzzer is disconnected, but it can be connected by 
        replacing the blue shorting block from its default horizontal position 
        to the one marked as 3/328P (Uno) or the one marked with 6/32U4 
        (Leonardo), respectively. The Zumo buzzer library can be used to play 
        more complicated songs.
  \item Battery monitor: One can monitor the battery voltage of the Zumo shield 
        using the A1 analog pin, but only if the jumper in the middle of the 
        board is connected by soldering a line between the A1 and Battery Level 
        holes. Otherwise, one can use A1 for other purposes.
  \item Inertial sensors: a combined accelerometer and magnetometer (compass) 
        as well as a gyroscope are connected via I2C. By default, these are 
        disconnected, but they can be forwarded to the SDA and SCL pins on the 
        bottom right of the board, close to LED13. Arduino versions older than 
        Uno R3 do not have these pins, which could allow us to connect them to 
        the Raspberry Pi, for example to achieve another external sensor which 
        could work with EKF2. On the Arduino Uno R3, the SDA and SCL pins are 
        duplicated to analog pins A4 and A5, respectively.
  \item I2C: pins 2 and 3 (Leonardo), or analog pins A4 and A5 (Uno), for 
        forwarding SDA and SCL respectively of a mounted Arduino. These pins 
        can have different purposes, including a connection with the 
        compass/gyroscope or a control line to the reflectance sensor array. 
        The sensor array can use a shorting block to determine which one to 
        use, which means it uses the pin that is not already in use for I2C.
  \item RX/TX: pins 0 and 1 for forwarding serial communication with a mounted 
        Arduino. One can also use any free pin with SoftwareSerial to use 
        a different UART RX/TX connection. For example, when using an Uno R3 
        with the battery level jumper disconnected, pins 6 and A1 can be freely 
        used for RX/TX.
  \item Reflectance sensor array: Six IR sensors and additional status LEDs. 
        Depending on how a shorting block is placed on top right of the sensor 
        array, one can enable or disable the sensors using a pin. If the block 
        is near to the Zumo shield then pin 2 is connected to LEDON, if it is 
        away from the shield with a bare pin visible, then pin A4 is connected 
        to LEDON. Send a high signal to the correct pin to enable the sensor 
        array, and a low signal to disable again; the visible red status LEDs 
        should show this. \\
        The IR sensors themselves are read from left to right, when facing the 
        forward driving direction with the shorting block in the back of the 
        array, as pins 4, A3, 11, A0, A2, and 5. \\
        All pins are forwarded via the front expansion board to the pins with 
        extenders placed on the Arduino mount. The sensor pins are both input 
        and output pins. In order to start a measurement, send a HIGH to all 
        pins and set them to be output pins for about 10 microseconds, then 
        make them input pins and set to LOW to disable pull-up resistors. The 
        pins will then send each send a pulse whose length determines the 
        blackness of the surface. A zero length pulse means completely white; 
        a pulse approaching infinite length (always 1) means completely black. 
        One should use a timeout of e.g. 2000 microseconds and consider all 
        pulse lengths above a certain threshold (e.g. 300 microseconds) to be 
        black.
\end{itemize}  

\section{References}

\begin{itemize}
  \item
    Zumo pin numbering: \url{https://www.pololu.com/docs/0J57/5}
  \item Arduino Uno pinout diagram: \url{http://i.stack.imgur.com/EsQpm.png}
  \item Arduino internal numbering details:
    \url{https://github.com/arduino/Arduino/blob/master/hardware/arduino/avr/variants/standard/pins_arduino.h#L50}:
  \item \url{https://www.pololu.com/picture/view/0J5572} labeled image of the 
        Zumo shield pins.
  \item \url{https://www.pololu.com/docs/0J57/3.c} explains how to solder the 
        battery level monitor jumper and how to read the battery voltage.
  \item Reflectance sensor array pin numbering:
    \url{https://github.com/pololu/zumo-shield/blob/master/ZumoReflectanceSensorArray/ZumoReflectanceSensorArray.h#L152}
  \item
    Documentation of the reflectance sensor array: 
    \url{https://www.pololu.com/docs/0J57/2.c}
  \item
    \url{https://github.com/pololu/zumo-shield/blob/master/QTRSensors/QTRSensors.cpp}
  \item Zumo schematic:
    \url{https://www.pololu.com/file/download/zumo-shield-v1_2-schematic.pdf?file_id=0J779}
  \item Arduino Uno schematic:
    \url{https://www.arduino.cc/en/uploads/Main/Arduino_Uno_Rev3-schematic.pdf}
  \item Arduino Uno: \url{https://www.arduino.cc/en/Main/ArduinoBoardUno}
\end{itemize}

\end{document}
